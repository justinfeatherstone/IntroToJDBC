\documentclass{article}
\usepackage{listings}
\usepackage{enumitem}
\usepackage{hyperref}

\title{JDBC Lab Analysis and Implementation Report}
\author{CS Database Course}
\date{\today}

\begin{document}
\maketitle

\section{Introduction}
This report analyzes the implementation of a JDBC-based database connectivity lab focusing on the Data Access Layer (DAL) pattern and three-tier architecture. The assignment required creating database connections, implementing various JDBC statements, and structuring code following software engineering best practices.

\section{Implementation Analysis}

\subsection{Data Manager Implementation (Requirement 1)}
The DataMgr class successfully implements the core requirements:
\begin{itemize}
    \item Centralized connection management through singleton pattern
    \item Connections to multiple databases (MealPlanning, ArcadeGames)
    \item Proper resource management with connection closing
    \item Robust error handling and logging
\end{itemize}

Code quality is high, with proper exception handling and logging using java.util.logging.

\subsection{Three-Layer Architecture (Requirement 2)}
The codebase successfully implements the three-layer architecture:

\subsubsection{Presentation Layer}
\begin{itemize}
    \item IntroToPresentationLayer.java handles user interaction
    \item Clean separation from business logic
    \item User input handling for database credentials
\end{itemize}

\subsubsection{Business Logic Layer}
Evidence of DTOs (Data Transfer Objects):
\begin{itemize}
    \item Recipe class
    \item Ingredient class
    \item ArcadeGame, Player, and Score classes
\end{itemize}

\subsubsection{Data Access Layer}
Well-structured DAL implementation:
\begin{itemize}
    \item DataMgr for connection management
    \item Separate DAL classes for different databases
    \item Proper resource cleanup
\end{itemize}

\subsection{Multiple DAL Implementation (Requirement 3)}
The ArcadeGamesDAL class demonstrates:
\begin{itemize}
    \item Statement usage for basic queries
    \item PreparedStatement for parameterized queries
    \item CallableStatement for stored procedures
    \item Consistent error handling and logging
\end{itemize}

\section{Technical Implementation Details}

\subsection{JDBC Statement Types}
The codebase demonstrates all three JDBC statement types:

1. Basic Statement:
\begin{itemize}
    \item Used in getAllRecipes() method
    \item Suitable for static queries
\end{itemize}

2. PreparedStatement:
\begin{itemize}
    \item Used in getIngredientsForRecipe()
    \item Prevents SQL injection
    \item Better performance for repeated execution
\end{itemize}

3. CallableStatement:
\begin{itemize}
    \item Implemented in getRecipesFromStoredProcedure()
    \item Proper parameter handling
    \item Stored procedure execution
\end{itemize}

\section{Areas for Improvement}
\begin{itemize}
    \item Connection pooling could be implemented for better performance
    \item Transaction management could be added
    \item More comprehensive error recovery mechanisms
    \item Unit tests could be added
\end{itemize}

\section{Learning Outcomes}
Through this lab, I gained practical experience with:
\begin{itemize}
    \item JDBC database connectivity
    \item Three-tier architecture implementation
    \item Different types of SQL statements
    \item Resource management in database applications
    \item Error handling and logging
    \item Software design patterns (Singleton, DAO)
\end{itemize}

\section{Conclusion}
The implementation successfully meets all core requirements while demonstrating good software engineering practices. The code is well-structured, maintainable, and follows proper separation of concerns. The experience provided valuable insights into real-world database application development and the importance of proper architectural design.

\end{document}